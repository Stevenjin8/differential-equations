\documentclass[11pt,letterpaper]{article}
\usepackage{titling} % must import before declaring date, author and title

\date{Sep 12, 2022}

\usepackage[utf8]{inputenc}
\usepackage[T1]{fontenc}
\usepackage{textcomp}
\usepackage[english]{babel}
\usepackage{url}
\usepackage{graphicx}
\usepackage{float}
\usepackage{booktabs}
\usepackage{enumitem}

\pdfminorversion=8

% Hide page number when page is empty
\usepackage{emptypage}
\usepackage{subcaption}
\usepackage{multicol}
\usepackage{xcolor}
% Other font I sometimes use.
% \usepackage{cmbright}

% Math stuff
\usepackage{amsmath, amsfonts, mathtools, amsthm, amssymb}
% Fancy script capitals
\usepackage{mathrsfs}
\usepackage{cancel}
% Bold math
\usepackage{bm}

% Command for short corrections
% Usage: 1+1=\correct{3}{2}

\definecolor{correct}{HTML}{009900}
\newcommand\correct[2]{\ensuremath{\:}{\color{red}{#1}}\ensuremath{\to }{\color{correct}{#2}}\ensuremath{\:}}
\newcommand\green[1]{{\color{correct}{#1}}}
\newcommand\ex[1]{\section*{Exercise #1}}

% horizontal rule
\newcommand\hr{
    \noindent\rule[0.5ex]{\linewidth}{0.5pt}
}

\newenvironment{cleanenum}
{ \begin{enumerate}[topsep=0pt]
    \setlength{\itemsep}{0pt}
    \setlength{\parskip}{0pt}
    \setlength{\parsep}{0pt}     }
{ \end{enumerate}  }
% hide parts
\newcommand\hide[1]{}

% si unitx
\usepackage{siunitx}
\sisetup{locale = US}

% Environments
\makeatother
% For box around Definition, Theorem, \ldots
\usepackage[linewidth=1pt, innerleftmargin=5pt, innerrightmargin=5pt]{mdframed}
\mdfsetup{skipabove=0.5em, skipbelow=0.5em}
\theoremstyle{definition}
\newcounter{lecture}
\setcounter{lecture}{0}
\newcounter{axiom}
\newtheorem{eg}{Example}[subsection]
\newtheorem*{notation}{Notation}
\newtheorem*{previouslyseen}{As previously seen}
\newtheorem*{remark}{Remark}
\newtheorem*{note}{Note}
\newtheorem*{problem}{Problem}
\newtheorem*{observe}{Observe}
\newtheorem*{property}{Property}
\newtheorem*{intuition}{Intuition}
\newtheorem*{axiom}{Axiom}
\newmdtheoremenv[nobreak=true]{prop}{Proposition}[subsection]
\newmdtheoremenv[nobreak=true]{theorem}{Theorem}[subsection]
\newmdtheoremenv[nobreak=true]{corollary}{Corollary}[subsection]
\newmdtheoremenv[nobreak=true]{definition}{Definition}[subsection]
\newmdtheoremenv[nobreak=true]{lemma}{Lemma}[subsection]
\newmdtheoremenv[nobreak=true]{conjecture}{Conjecture}[subsection]


\renewcommand{\thesubsection}{\arabic{lecture}}
% End example and intermezzo environments with a small diamond (just like proof
% environments end with a small square)
\usepackage{etoolbox}
\AtEndEnvironment{vb}{\null\hfill$\diamond$}%
\AtEndEnvironment{intermezzo}{\null\hfill$\diamond$}%
\AtEndEnvironment{opmerking}{\null\hfill$\diamond$}%

% Fix some spacing
% http://tex.stackexchange.com/questions/22119/how-can-i-change-the-spacing-before-theorems-with-amsthm
\makeatletter
\def\thm@space@setup{%
  \thm@preskip=\parskip \thm@postskip=0pt
}


% \lecture starts a new lecture (les in dutch)
%
% Usage:
% \lecture{1}{di 12 feb 2019 16:00}{Inleiding}
%
% This adds a section heading with the number / title of the lecture and a
% margin paragraph with the date.

\def\@lecture{Lecture \arabic{lecture}}%
\usepackage{xifthen}
\newcommand{\lecture}[1]{
  \stepcounter{lecture}
	\def\@date{#1}
    \subsection*{\@lecture}
	\setcounter{definition}{0}
	\setcounter{eg}{0}
	\setcounter{lemma}{0}
	\setcounter{prop}{0}
	\setcounter{conjecture}{0}
	\setcounter{corollary}{0}
	\setcounter{theorem}{0}
	\setcounter{axiom}{0}
}

% These are the fancy headers
\usepackage{fancyhdr}
\pagestyle{fancy}

% LE: left even
% RO: right odd
% CE, CO: center even, center odd
% My name for when I print my lecture notes to use for an open book exam.
% \fancyhead[LE,RO]{Gilles Castel}

\setlength{\headheight}{27.11469pt}
\fancyhead[L]{\@lecture\\ \@date}
\fancyhead[R]{Steven Jin}
%
\makeatother
%
% Todonotes and inline notes in fancy boxes
\usepackage{tcolorbox}
%
% Make boxes breakable
\tcbuselibrary{breakable}
%
% Figure support as explained in my blog post.
\usepackage{import}
\usepackage{xifthen}
\usepackage{pdfpages}
\usepackage{transparent}
\newcommand{\incfig}[1]{%
    \def\svgwidth{\columnwidth}
    \import{./figures/}{#1.pdf_tex}
}
% Fix some stuff
%http://tex.stackexchange.com/questions/76273/multiple-pdfs-with-page-group-included-in-a-single-page-warning
\pdfsuppresswarningpagegroup=1




\renewcommand\ex[1]{\section*{Exercise #1}}

\title{Homework 1}

\begin{document}
\maketitle
\setcounter{exercise_section}{1}

\pagebreak

\ex{4}
First, we find
\[
  \int x\cos(2x) dx
\]
using integration by parts. Let $f(x) = x$ and $g'(x) = \cos(2x)$.
The antiderivative of $\cos(2x)$ is $\frac12\sin(2x)$ and $f'(x) = 1$
Applying the formula for integration by parts, we get
\begin{align*}
  \int x\cos(2x) dx
  &= f(x)g(x) - \int f'(x)g(x) dx \\
  &= x \frac12 \sin(2x) - \int 1 \cdot \frac12 \sin(2x) dx \\
  &= x \frac12 \sin(2x) + \frac14 \cos(2x) dx + C.
\end{align*}

Now, we find the integral for
\[
  \int x^2 \sin(2x)dx.
\]
Once again, we use integration by parts.
Let $f(x) = x^2$ and $g'(x) = \sin(2x)$.
Using trig antiderivatives, we have $g(x) = -\frac12 \cos(2x)$.
Applying the integration by parts formula and our earlier calculation, we get
\begin{align*}
  \int x^2 \sin(2x)dx
  &= f(x)g(x) - \int f'(x)g(x) dx \\
  &= -\frac12x^2\cos(2x) + \int x \cos(2x) dx \\
  &= -\frac12x^2\cos(2x) + x \frac12 \sin(2x) + \frac14 \cos(2x) dx + C \\
  &= \frac14(1 - 2x^2) \cos(2x) + x \frac12 \sin(2x) + C.
\end{align*}

\ex{10}

We use integration for parts. Let
\begin{align*}
  f'(x) &= x^{\frac25} \\
  g(x) &= \ln x.
\end{align*}
We can choose $f(x) = \frac57 x^{7/5}$ using the Power Law and find $g'(x) = 1/x$.
Applying the integration by parts formula, we find that
\begin{align*}
  \int x^{\frac25} \ln x dx
  &= f(x)g(x) - \int f'(x)g(x) dx \\
  &= \frac57 x^{\frac75} \ln x - \int \frac1x \frac57 x^{\frac75} dx \\
  &= \frac57 x^{\frac75} \ln x - \frac57 \int \frac1x x^{\frac75} dx \\
  &= \frac57 x^{\frac75} \ln x - \frac57 \int x^{\frac25} dx \\
  &= \frac57 x^{\frac75} \ln x - \frac57 \frac57 x^{\frac75} + C \text{   (Power Rule)} \\
  &= \frac57 x^{\frac75} \ln x - \frac{ 25 }{ 49 } x^{\frac75} + C. \\
  &= \left(\frac57 \ln x - \frac{ 25 }{ 49 }\right) x^{\frac75} + C.
\end{align*}

Now, we find the definite integral:
\begin{align*}
  \int\limits_{1}^{2} x^{2/5}\ln x dx
  &= \left[\left(\frac57 \ln x - \frac{ 25 }{ 49 }\right) x^{\frac75}\right]_1^2 \\
  &= \left(\frac57 \ln 2 - \frac{ 25 }{ 49 }\right) 2^{\frac75}
  - \left(\left(\frac57 \ln 1 - \frac{ 25 }{ 49 }\right) 1^{\frac75}\right) \\
  &= \left(\frac57 \ln 2 - \frac{ 25 }{ 49 }\right) 2^{\frac75}
  + \frac{ 25 }{ 49 } \\
  &\approx 0.4703577
\end{align*}


\ex{11}
First, we work with the integrand. We can rewrite
\begin{align*}
  \frac{ x + 1 }{ x^2 - 4 } &= \frac{ x + 1 }{ (x + 2)(x - 2) } \\
  &= \frac{ A(x - 2) + B(x + 2) }{ (x + 2)(x - 2) } \\
  &= \frac{ A }{ x + 2 } + \frac{ B }{ x - 2 } 
\end{align*}
for some $A, B \in \mathbb{R}$. Setting $A = 1/4$ and $B = 3/4$, we see that
\[
  \frac{ (x - 2)/4  + 3(x + 2)/4}{  (x + 2)(x - 2)  }
  = \frac{ x/4 + 3x/4 - 1/2 + 3/2}{  (x + 2)(x - 2)  }
  = \frac{ x + 1 }{ (x + 2)(x - 2) }. \\
\]
Thus, we can write our integral as
\begin{align*}
  \int \frac{ x + 1 }{ x^2 - 4 } dx
  &= \int \frac{ 1/4 }{ x - 2 } + \frac{ 3/4 }{ x + 2 } dx \\
  &= \frac14 \int \frac{ 1 }{ x - 2 } dx + \frac34 \int \frac{ 1 }{ x + 2 } dx \\
  & = \frac{ \ln (x - 2) }{ 4 } + \frac{ 3 \ln (x + 2) }{ 4 } + C.
\end{align*}


\end{document}
