\documentclass{article}[11, letterpaper]
\author{Steven Jin Xuan}
\date{Fall 2022}
\title{Differential Equations}

\usepackage[utf8]{inputenc}
\usepackage[T1]{fontenc}
\usepackage{textcomp}
\usepackage[english]{babel}
\usepackage{url}
\usepackage{graphicx}
\usepackage{float}
\usepackage{booktabs}
\usepackage{enumitem}

\pdfminorversion=8

% Hide page number when page is empty
\usepackage{emptypage}
\usepackage{subcaption}
\usepackage{multicol}
\usepackage{xcolor}
% Other font I sometimes use.
% \usepackage{cmbright}

% Math stuff
\usepackage{amsmath, amsfonts, mathtools, amsthm, amssymb}
% Fancy script capitals
\usepackage{mathrsfs}
\usepackage{cancel}
% Bold math
\usepackage{bm}

% Command for short corrections
% Usage: 1+1=\correct{3}{2}

\definecolor{correct}{HTML}{009900}
\newcommand\correct[2]{\ensuremath{\:}{\color{red}{#1}}\ensuremath{\to }{\color{correct}{#2}}\ensuremath{\:}}
\newcommand\green[1]{{\color{correct}{#1}}}
\newcommand\ex[1]{\section*{Exercise #1}}

% horizontal rule
\newcommand\hr{
    \noindent\rule[0.5ex]{\linewidth}{0.5pt}
}

\newenvironment{cleanenum}
{ \begin{enumerate}[topsep=0pt]
    \setlength{\itemsep}{0pt}
    \setlength{\parskip}{0pt}
    \setlength{\parsep}{0pt}     }
{ \end{enumerate}  }
% hide parts
\newcommand\hide[1]{}

% si unitx
\usepackage{siunitx}
\sisetup{locale = US}

% Environments
\makeatother
% For box around Definition, Theorem, \ldots
\usepackage[linewidth=1pt, innerleftmargin=5pt, innerrightmargin=5pt]{mdframed}
\mdfsetup{skipabove=0.5em, skipbelow=0.5em}
\theoremstyle{definition}
\newcounter{lecture}
\setcounter{lecture}{0}
\newcounter{axiom}
\newtheorem{eg}{Example}[subsection]
\newtheorem*{notation}{Notation}
\newtheorem*{previouslyseen}{As previously seen}
\newtheorem*{remark}{Remark}
\newtheorem*{note}{Note}
\newtheorem*{problem}{Problem}
\newtheorem*{observe}{Observe}
\newtheorem*{property}{Property}
\newtheorem*{intuition}{Intuition}
\newtheorem*{axiom}{Axiom}
\newmdtheoremenv[nobreak=true]{prop}{Proposition}[subsection]
\newmdtheoremenv[nobreak=true]{theorem}{Theorem}[subsection]
\newmdtheoremenv[nobreak=true]{corollary}{Corollary}[subsection]
\newmdtheoremenv[nobreak=true]{definition}{Definition}[subsection]
\newmdtheoremenv[nobreak=true]{lemma}{Lemma}[subsection]
\newmdtheoremenv[nobreak=true]{conjecture}{Conjecture}[subsection]


\renewcommand{\thesubsection}{\arabic{lecture}}
% End example and intermezzo environments with a small diamond (just like proof
% environments end with a small square)
\usepackage{etoolbox}
\AtEndEnvironment{vb}{\null\hfill$\diamond$}%
\AtEndEnvironment{intermezzo}{\null\hfill$\diamond$}%
\AtEndEnvironment{opmerking}{\null\hfill$\diamond$}%

% Fix some spacing
% http://tex.stackexchange.com/questions/22119/how-can-i-change-the-spacing-before-theorems-with-amsthm
\makeatletter
\def\thm@space@setup{%
  \thm@preskip=\parskip \thm@postskip=0pt
}


% \lecture starts a new lecture (les in dutch)
%
% Usage:
% \lecture{1}{di 12 feb 2019 16:00}{Inleiding}
%
% This adds a section heading with the number / title of the lecture and a
% margin paragraph with the date.

\def\@lecture{Lecture \arabic{lecture}}%
\usepackage{xifthen}
\newcommand{\lecture}[1]{
  \stepcounter{lecture}
	\def\@date{#1}
    \subsection*{\@lecture}
	\setcounter{definition}{0}
	\setcounter{eg}{0}
	\setcounter{lemma}{0}
	\setcounter{prop}{0}
	\setcounter{conjecture}{0}
	\setcounter{corollary}{0}
	\setcounter{theorem}{0}
	\setcounter{axiom}{0}
}

% These are the fancy headers
\usepackage{fancyhdr}
\pagestyle{fancy}

% LE: left even
% RO: right odd
% CE, CO: center even, center odd
% My name for when I print my lecture notes to use for an open book exam.
% \fancyhead[LE,RO]{Gilles Castel}

\setlength{\headheight}{27.11469pt}
\fancyhead[L]{\@lecture\\ \@date}
\fancyhead[R]{Steven Jin}
%
\makeatother
%
% Todonotes and inline notes in fancy boxes
\usepackage{tcolorbox}
%
% Make boxes breakable
\tcbuselibrary{breakable}
%
% Figure support as explained in my blog post.
\usepackage{import}
\usepackage{xifthen}
\usepackage{pdfpages}
\usepackage{transparent}
\newcommand{\incfig}[1]{%
    \def\svgwidth{\columnwidth}
    \import{./figures/}{#1.pdf_tex}
}
% Fix some stuff
%http://tex.stackexchange.com/questions/76273/multiple-pdfs-with-page-group-included-in-a-single-page-warning
\pdfsuppresswarningpagegroup=1



\begin{document}

\maketitle

\pagebreak
\lecture{Sep 12, 2022}

The question of diffeqs is about reversing derivatives.
Usually, we know about movement in terms of its velocity, not its position.

\begin{definition}[Differential Equation]
  An equation tells us about a function in terms of its values and derivatives.
  In other words, an identity over some region that connects a function with its derivatives.
\end{definition}

\begin{eg}
  Given
  \[
    f'' = 2f' + f^2
  \]
  what can we say about $f$?
  Does it move periodically as the planets around the sun?
  Does it have a limit? Is it bounded?
\end{eg}

\begin{eg}
  Let
  \[
    y = f(t)
  \]
  with
  \[
    y' = 12y
  \]
  for all $t \in \mathbb{R}$.
  Let's guess that
  \[
    \phi(t) = 9 e^{12t}.
  \]
  Thus,
  \[
    \phi'(t) = 12 \cdot 9 \cdot e^{12t} = 12 \phi(t).
  \]
\end{eg}

But how did we find the above solution?
How do we find explicit solutions?
Sometimes, we can find explicit solutions.
But for most interesting diffeqs, we have to take a more qualitative approach.
I.e., find properties of diffeqs.
Finally, we can get close to the explicit solution via numerical approximation.
How well can we approximate? Are error bounds guaranteed?

Another issue is that the same differential equation can have many solutions.

\begin{theorem}
  Say we want to solve
  \[
    f'(t) = g(t).
  \]
  The solution is
  \[
    f(t) = \int g(t) dt
  \]
  if $g$ has an integral.
  But what if we don't know how to integrate $g$?
\end{theorem}

\begin{eg}
  Let
  \[
    f'(t) = f(t), f(0) = 3
  \]
  Differentiating, we get $f'/f = 1$.
  Integrating both sides, we get
  \[
    \int \frac{ f'(t) }{ f(t) } dt = t + C.
  \]
  However, we know that
  \[
    \int \frac{ f'(t) }{ f(t) } dt = \ln f(t) + C.
  \]
  Thus,
  \[
    \ln f(t) = t + C \implies f(t) = e^{t + C}.
  \]
  Plugging this into our initial condition, we get $C = \ln3$.
  So,
  \[
    f(t) = 3e^{t}.
  \]
\end{eg}

\lecture{Sep 14, 2022}

First, some review and history.
Matrix algebra and differential equations are actually mathematically equivalent.
The solution to differential equations are functions, not numbers.
Focus will be on ODE.

\begin{definition}[Order]
  The highest derivative in a differential equation.
\end{definition}

The solution to a differential equation can be expressed as
\[
  F(t, y', y'', \ldots, y^{(n)}) = 0
\]
over some interval $I$.
Thus, a candidate function $\phi$ must be differential $n$ times.
Checking a proposed solution is easier. Coming up with one is harder.

\begin{theorem}
  If $f$ and $g$ have the same derivative then
  \[
    f(x) - g(x) = C
  \]
  for all $x$.
\end{theorem}
\begin{proof}
  Let $h = f - g$.
  It follows that $h' = (f - g)' = f' - g' = 0$.
  Thus, $h$ must be a constant because if it weren't, then by the Mean Value Theorem, the derivative of $h$ wouldn't always be 0.
  Thus,
  \[
    f(x) = g(x) + C
  \]
  for some $C \in \mathbb{R}$.
\end{proof}


\end{document}
